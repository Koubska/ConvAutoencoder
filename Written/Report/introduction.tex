\section{Introduction}
\label{sec:intro}
Pain is a multifaceted sensory and emotional sensation.
It serves as the body's alarm system, warning us of impending damage or harm and encouraging preventive activities.
However, when pain persists for an extended time, it can develop into a debilitating condition with adverse effects on a person's quality of life.
Neuroscience, psychology, and medicine are just a few of the numerous fields involved in the study of pain \cite{robins2016pain,eccleston2013psychological}.
In recent years, there has been an increase in interest in using neuroimaging techniques to study the brain mechanisms underlying pain perception and the development of chronic pain conditions \cite{schmidt2015neuroimaging, da2019neuroimaging}.
The technique of Microneurography (MNG) records the activity of single C-nociceptors in patients with neuropathic pain, providing an approach to investigate the underlying mechanisms of pathological discharges in these peripheral nerve fibers \cite{vallbo2004microneurography,vallbo2018microneurography}.
MNG employs a thin electrode placed within a peripheral nerve to record action potentials from neighboring fibers, but sorting these signals to single nerve fibers is necessary to understand neural coding. 
However, standard shape-based sorting algorithms, successful in in-vitro data, cannot be reliably used in MNG due to low signal-to-noise ratio and high variability in spike shapes. %TODO Reference
Thus, there are two problems: 
\begin{enumerate*}
	\item Spike Detection and
	\item Spike Sorting.
\end{enumerate*}
In this report, we are going to introduce \textit{Convolution Autoencoders (CAE)} to help with the aforementioned problems.
A CAE is a special type of  autoencoder, which is commonly used in computer vision tasks to detect features in images, as they can detect patterns in input data \cite{geron2017hands,geng2016human,cheng2018deep}.
Because a CAE is capable of detecting the most important parts of the input data, the hope is that spikes get detected and appropriately compressed (Encoder) and decompressed (Decoder).
Further, spike detection and further analysis can be done on the encoded data.
For that to work reasonably the features in the signal in form of the spikes have to be detected and preserved by a used CAE.
The workflow of this project can be seen in \Cref{fig:workflow}.
\begin{figure}[h]
	\centering
	\begin{tikzpicture}
		\node[flow, fill=green!20] at (1,0) (rawData) {Raw Data};
		\node[flow, fill=green!20] at (4,0) (cleanup) {Cleanup};
		\node[flow, fill=green!20] at (6,0) (normalize) {Normalizing};
		\node[flow, fill=blue!20] at (6,-1) (encode) {Encode};
		\node[flow, fill=purple!20] at (4,-1) (analyze) {Analyze};
		\node[flow, fill=pink!20] at (2,-1) (decode) {Decode};
		\node[flow, fill=black!20] at (0,-1) (compare) {Compare};
		\path[connector] (rawData) --(cleanup);
		\path[connector] (cleanup) -- (normalize);
		\path[connector] (normalize) -- (encode);
		\path[connector] (encode) -- (analyze);
		\path[connector] (analyze) -- (decode);
		\path[connector] (decode) -- (compare);
		\draw[thick, green, draw] ($(cleanup.north west) + (-0.2,0.2)$) rectangle ($(normalize.south east) + (0.2,-0.2)$);
		\node[] at (5,0.7) (preproc) {Preprocessing};
	\end{tikzpicture}
	\label{fig:workflow}
	\caption{Workflow for using CAEs to compress and extract features of neural data.}
\end{figure}
\FloatBarrier