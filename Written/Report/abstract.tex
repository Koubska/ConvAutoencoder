\begin{abstract}
    This report discusses the use of Convolution Autoencoders (CAE) to address two main problems in the field of pain research: Spike Detection and Spike Sorting in Microneurography (MNG) data. 
    MNG records the activity of single C-nociceptors in patients with neuropathic pain, providing insights into the mechanisms of pathological discharges in these peripheral nerve fibers. 
    However, the low signal-to-noise ratio and high variability in spike shapes in MNG data make it challenging to accurately detect and sort spikes. 
    CAEs are a type of autoencoder commonly used in computer vision tasks to detect features in images. 
    The report proposes the use of CAEs to detect and compress spikes in MNG data, which can be further analyzed to understand the neural coding underlying pain perception.
\end{abstract}